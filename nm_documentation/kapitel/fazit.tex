\chapter{Fazit und Ausblick}

Im Rahmen dieser Arbeit wurde sich mit den Problemen der Erkundung und Navigation im RL Kontext beschäftigt. Hierzu wurde eine Erweiterung für den Schreiboperators der Neural Map entwickelt. Der ursprünglichen Schreiboperator beschreibt lediglich die Speicherposition, die der aktuellen Position des Agenten entspricht. Dies wurde als Limitierung der Neural Map ausgemacht. Somit beschreibt der erweiterte Schreiboperator eine eine weitere bzw. zusätzliche Speicherposition der Neural Map. Den Ansatz für die Auswahl dieser zusätzlichen Speicherposition bildet die Blickrichtung des Agenten. In dieser Richtung enthält die Beobachtung des Agenten für gewöhnlich auch Informationen die nicht an der aktuellen Position verortet werden können. Somit sollen diese an die nächste gültige Speicherposition in Blickrichtung geschrieben werden. Dass die Erweiterung des Schreiboperators die Neural Map verbessert, konnte im Rahmen mehrerer 2D-Experimente und eines 3D-Experiments ausführlich gezeigt werden. Dabei bestand die Aufgabe des Agenten darin in verschiedenen Umgebungen nacheinander mehrere Ziele in einer bestimmten Reihenfolge zu finden. Hierbei benötigte die Neural Map mit der Erweiterung des Schreiboperators im Durchschnitt grundsätzlich weniger Schritte als die ursprüngliche Variante. Darüber hinaus wurden in einem seperaten Speichertest gezeigt, dass die Neural Map in ihrem Speicher eine fehlerfreie Karte der Umgebung generieren kann. Innerhalb der Diskussion konnte auch für dieses Experiment eine Verbesserung durch die Erweiterung des Schreiboperators begründet werden.



Innerhalb der Diskussion wurde vermutet, dass der Agent seine eigene Position nur unzureichend lokalisieren kann und infolgedessen keine zielgerichtete Navigation vornehmen kann. Somit wäre es von Interesse, diese Fähigkeit in einem eigens dafür entwickelten Experiment zu untersuchen. Dabei sollte insbesondere eine Umgebung verwendet werden, bei der eine Lokalisierung aufgrund der aktuellen Beobachtung nicht eindeutig möglich ist. Der hindernisfreie Raum in dieser Arbeit könnte dazu als Ansatz verwendet werden. Um im inneren dieses Raums seine Position korrekt zu bestimmen, muss der Agent von seiner letzten bekannten Position seinen zurückgelegten Weg berücksichtigen. Eine weiterer interessanter Aspekt ist die Erstellung der Karte im Speicher der Neural Map. Das dies prinzipiell möglich ist, konnte innerhalb des Speichertests gezeigt werden. Allerdings liegt dieser Test außerhalb des eigentlichen RL Kontext und die zur Erstellung der Karte benötigte Schrittanzahl ist ziemlich hoch. Es stellt sich jedoch die Frage, inwiefern sich während eines normalen RL Trainingsprozesses innerhalb des Lese- und Schreiboperators Strukturen ausbilden, die eine generelle Kartenerstellung ermöglichen. Dies könnte ebenfalls in einem weiteren Experiment untersucht werden.
