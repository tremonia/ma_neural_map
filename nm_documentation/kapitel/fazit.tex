\chapter{Fazit und Ausblick}

Im Rahmen dieser Arbeit wird sich mit den Problemen der Erkundung und Navigation im RL Kontext beschäftigt. Hierzu wird eine Erweiterung für den Schreiboperator der Neural Map entwickelt. Der ursprüngliche Schreiboperator beschreibt lediglich die Speicherposition, die der aktuellen Position des Agenten entspricht. Dies wird als Limitierung der Neural Map ausgemacht. Somit beschreibt der erweiterte Schreiboperator eine weitere bzw. zusätzliche Speicherposition der Neural Map. Den Ansatz für die Auswahl dieser zusätzlichen Speicherposition bildet die Blickrichtung des Agenten. In dieser Richtung enthält die Beobachtung des Agenten für gewöhnlich auch Informationen, die nicht an der aktuellen Position verortet werden können. Somit sollen diese an die nächste gültige Speicherposition in Blickrichtung geschrieben werden. Dass die Erweiterung des Schreiboperators die Neural Map verbessert, konnte im Rahmen mehrerer 2D-Experimente und eines 3D-Experiments ausführlich gezeigt werden. Dabei bestand die Aufgabe des Agenten darin, in verschiedenen Umgebungen nacheinander mehrere Ziele in einer bestimmten Reihenfolge zu finden. Hierbei benötigte die Neural Map, mit der Erweiterung des Schreiboperators, grundsätzlich im Durchschnitt weniger Schritte als die ursprüngliche Variante. Diese Ergebnisse können allerdings nur sehr bedingt mit einer Verbesserung der Navigationsfähigkeit in Verbindung gebracht werden. Es lassen sich einzig Ansätze dafür finden, dass die Neural Map das Explorationsverhalten des Agenten begünstigt. Die Erweiterung des Schreiboperators hingegen scheint höchstens die ersten Schritte der Exploration positiv zu beeinflussen. Somit bleibt fraglich, ob die Navigationsfähigkeit insgesamt durch die Erweiterung des Schreiboperators verbessert wird. Die Experimente dieser Arbeit bzw. die darin verwendeten Problemstellungen geben darüber nur sehr eingeschränkt Auskunft. Darüber hinaus wird in einem separaten Speichertest gezeigt, dass die Neural Map in ihrem Speicher eine fehlerfreie Karte der Umgebung generieren kann. Allerdings benötigt sie dafür eine sehr hohe Anzahl von Trainingssschritten. Innerhalb der Diskussion kann für dieses Experiment begründet werden, dass durch die Erweiterung des Schreiboperators eine Karte der Umgebung von höherer Qualität erzeugt werden kann.

Ebenfalls wird in der Diskussion vermutet, dass der Agent seine eigene Position nur unzureichend lokalisieren kann und infolgedessen keine zielgerichtete Navigation vornehmen kann. Somit wäre es von Interesse, diese Fähigkeit in einem eigens dafür entwickelten Experiment zu untersuchen. Dabei sollte insbesondere eine Umgebung verwendet werden, bei der eine Lokalisierung aufgrund der aktuellen Beobachtung nicht eindeutig möglich ist. Der hindernisfreie Raum in dieser Arbeit könnte dazu als Ansatz verwendet werden. Darüber hinaus muss die Umgebung so verändert werden, dass das Ziel des Agenten eindeutig ist und sichergestellt ist, dass der Agent diese Information auch besitzt. Insgesamt muss ein Szenario geschaffen werden, in dem die Lokalisierung der eigenen Position die letzte fehlende Information des Agenten ist, um eine zielgerichtete Navigation zu ermöglichen.

Eine weiterer interessanter Aspekt ist die Erstellung der Karte im Speicher der Neural Map. Das dies prinzipiell möglich ist, konnte innerhalb des Speichertests gezeigt werden. Allerdings liegt dieser Test außerhalb des eigentlichen RL Kontext und die zur Erstellung der Karte benötigte Schrittanzahl ist ziemlich hoch. Es stellt sich jedoch die Frage, inwiefern sich während eines normalen RL Trainingsprozesses innerhalb des Lese- und Schreiboperators Strukturen ausbilden, die eine generelle Kartenerstellung ermöglichen. Dies könnte ebenfalls in einem weiteren Experiment untersucht werden. Dazu durchläuft der Agent zunächst eine ganz normale RL Trainingsprozedur für eine Problemstellung, die eine Karte erfordert. Anschließend bekommt der Agent speziell ausgewählte Observationen präsentiert. Diese Observationen sollten wahlweise sehr markante Charakteristika der Umgebung enthalten oder keinerlei relevante Informationen. Nun kann jede Aktualisierung der Karte in Bezug gesetzt werden zur jeweiligen Observation. Dies ist jedoch die größte Schwierigkeit des Experiments, da der Zusammenhang zwischen der Observation und dem, was in die Karte geschrieben wird, in der Regel sehr komplex ist. Eine erste Lösung hierfür könnte in der Verwendung gleicher Observationen an unterschiedlichen Positionen bestehen. Diese gleichen Observationen sollten auch zu gleichen Einträgen in der Karte führen.
